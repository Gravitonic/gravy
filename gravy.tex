\makeatletter
\declare@file@substitution{article.cls}{scrartcl.cls}
\makeatother

\documentclass{l3doc}

% l3doc sets \parindent to 2em
% restore to scrartcl value of 1em
\setlength\parindent{1em}

\usepackage[nodate]{gravy}

\usepackage{lipsum}


\title{The \pkg{gravy} package}
\author{Dylan Yang}

\begin{document}

\maketitle

\tableofcontents

\clearpage

\section{Introduction}

The \pkg{gravy} package is a collection of my personal \LaTeX\ styles and commands.

\section{Usage}

Typically, \pkg{gravy} should be loaded as a package (i.e. with \cs{usepackage}) after specifying a KOMA-Script document class. Note that \pkg{gravy} currently only supports pdf\TeX.

For convenience, the \cls{gravyartcl} and \cls{gravyreprt} document classes are also available. These classes simply load the corresponding KOMA-Script document class and the \pkg{gravy} package.

\pkg{gravy} has two options. The \texttt{nodate} option disables the printing of the date when using \cs{maketitle}. The \texttt{minted} option loads custom styles for use with the \pkg{minted} package. Options can be passed to the \pkg{gravy} package or document classes.

\section{Typography}

\subsection{Fonts}

The serif font is Linux Libertine, while the sans serif font is \textsf{Linux Biolinum}. The monospace font is \texttt{Fira Mono}. The math font is the libertine math font provided by the \pkg{newtxmath} package.

Here is a demonstration of some of the supported ligatures and kerning.

\begin{center}
  \huge{ff fi fl ffi ffl}

  \vspace{0.5em}

  \huge{\textit{ff fi fl ffi ffl}}

  \vspace{0.5em}

  \huge{AV WA Ay Kw Te fo w.}
\end{center}

Note that additional ligatures such as `Th' and `Qu' are not currently supported given the use of \texttt{T1} font encoding; see \href{https://tex.stackexchange.com/q/373440}{this question} on the TeX StackExchange. Additionally, the current monospace font, Fira Mono, does not support certain characters such as `\textbackslash' when using \texttt{OT1} encoding instead. Future versions may add an option for Xe\LaTeX\ or Lua\LaTeX\ compatibility, which support OpenType fonts and should avoid these issues.

\subsection{Line Width}

\newlength{\alphabetwidth}
\settowidth{\alphabetwidth}{abcdefghijklmnopqrstuvwxyz}

One lowercase alphabet is \the\alphabetwidth\ in width; the line width is \the\linewidth. This should follow the general typographic advice that the line width should be about two to three lowercase alphabets.

\subsection{Fonts and Styles}

\begin{itemize}
  \item The quick brown fox jumps over the lazy dog
  \item \textbf{The quick brown fox jumps over the lazy dog}
  \item \emph{The quick brown fox jumps over the lazy dog}
  \item \textbf{\emph{The quick brown fox jumps over the lazy dog}}
  \item \textsc{The quick brown fox jumps over the lazy dog}
  \item \textsc{\textbf{The quick brown fox jumps over the lazy dog}}
  \item \textsc{\emph{The quick brown fox jumps over the lazy dog}}
  \item \textsc{\emph{\textbf{The quick brown fox jumps over the lazy dog}}}
  \item \textsf{The quick brown fox jumps over the lazy dog}
  \item \textsf{\textbf{The quick brown fox jumps over the lazy dog}}
  \item \textsf{\emph{The quick brown fox jumps over the lazy dog}}
  \item \textsf{\emph{\textbf{The quick brown fox jumps over the lazy dog}}}
  \item \texttt{The quick brown fox jumps over the lazy dog}
  \item \texttt{\textbf{The quick brown fox jumps over the lazy dog}}
  \item \texttt{\emph{The quick brown fox jumps over the lazy dog}}
  \item \texttt{\textbf{\emph{The quick brown fox jumps over the lazy dog}}}
\end{itemize}

\section{Commands}

\subsection{Delimiter Sizing}

\begin{function}{\auto}
  \begin{syntax}
    \cs{auto} [\meta{opening delimiter}\meta{closing delimiter}] \meta{delimited expression}
  \end{syntax}

  When typesetting math equations, it is common to use \cs{left} and \cs{right} to automatically size delimiters, but this can become cumbersome in more complicated use cases. \cs{auto} automatically detects the next pair of delimiters and sizes them appropriately. For instance, "\auto(\frac{1}{2})" produces \[ \auto(\frac{1}{2}). \]

  The delimiters that can be automatically detected are: \begin{itemize}
    \item "(" and ")"
    \item "\lparen" and "\rparen"
    \item "[" and "]"
    \item "\lbrack" and "\rbrack"
    \item "\{" and "\}"
    \item "\lbrace" and "\rbrace"
    \item "<" and ">"
    \item "\langle" and "\rangle"
    \item "|" and "|"
    \item "\vert" and "\vert"
    \item "\lvert" and "\rvert"
    \item "\|" and "\|"
    \item "\Vert" and "\Vert"
    \item "\lVert" and "\rVert"
  \end{itemize}

  Note that under the hood, \cs{auto} only looks for the next supported opening delimiter and assumes that the closing delimiter based on that, so it only supports matching pairs of delimiters. For instance, "\auto.\dv{f}{t}\rvert_{t=0}" does not work. Instead, to handle arbitrary pairs of delimiters, \cs{auto} can also take one optional argument, which should be two tokens representing the opening and closing delimiters. "\auto[.\rvert].\dv{f}{t}\rvert_{t=0}" works instead, producing \[ \auto[.\rvert] .\dv{f}{t}\rvert_{t=0}. \]
\end{function}

\subsection{Derivatives}

\begin{function}{\dd}
  \begin{syntax}
    \cs{dd} \marg{variable}
  \end{syntax}

  The \cs{dd} command is used to typeset the differential operator. It produces an upright `d' with math operator spacing preceding it, as shown in the example below:
  \[\int x \dd x.\]
\end{function}

\begin{function}{\deriv, \dv, \pderiv, \pdv}
  \begin{syntax}
    \cs{deriv}  * \oarg{power} \marg{numerator} \marg{denominator}
    \cs{dv}     * \oarg{power} \marg{numerator} \marg{denominator}
    \cs{pderiv} * \oarg{power} \marg{numerator} \marg{denominator}
    \cs{pdv}    * \oarg{power} \marg{numerator} \marg{denominator}
  \end{syntax}

  Derivatives can be typeset using the \cs{deriv} and \cs{dv} commands (which are equivalent), while partial derivatives can be typeset using the \cs{pderiv} and \cs{pdv} commands (which are also equivalent). They take two mandatory arguments, the \meta{numerator} and \meta{denominator}, as well as an optional argument \meta{power} which specifies the power to which the derivative is raised and an optional star to produce an inline fraction instead of a display fraction.
\end{function}

\subsection{Semantic Delimiters}

\begin{function}{\abs, \norm, \set, \floor, \ceil}
  \begin{syntax}
    \cs{abs} * \oarg{size} \marg{expression}
  \end{syntax}

  \cs{abs}, \cs{norm}, \cs{set}, \cs{floor}, and \cs{ceil} offer more semantic names to wrap an \meta{expression} in the appropriate delimiters. By default, the size of the delimiters is automatically determined using \cs{left} and \cs{right}, but this can be overridden by specifying a specific size using the optional \meta{size} argument or by including an optional star to avoid sizing the delimiters.

  Note that \cs{set} also adds a \verb|\,| space after the opening brace and before the closing brace. The set on the left below uses \cs{set} while the set on the right does not: \[ \set{1, 2, 3} \quad \{1, 2, 3\} \]
\end{function}

\begin{function}{\innerp}
  \begin{syntax}
    \cs{innerp} * \oarg{size} \marg{expression} \marg{expression}
  \end{syntax}

  \cs{innerp} typesets inner products, e.g. $\innerp{3}{4}$. It likewise has automatically sized delimiters by default which can be manually sized with an optional argument, but it takes two arguments instead of one.
\end{function}

\begin{function}{\conj}
  \begin{syntax}
    \cs{conj} \marg{expression}
  \end{syntax}

  \cs{conj} is an alias for \cs{overline} that also works outside of math mode. It is a more semantic name for the (complex) conjugate.
\end{function}

\subsection{Sets}

\begin{function}{\given, \suchthat}
  \begin{syntax}
    \cs{given}
    \cs{suchthat}
  \end{syntax}

  The \cs{given} command (alias: \cs{suchthat}) is used to specify the condition for a set. If it is preceded by a (yet-to-be-closed) \cs{left}, then it is equivalent to "\middle\vert" with math relation spacing on both sides; otherwise, it is equivalent to "\mid".
\end{function}

\begin{function}{\RR, \CC, \NN, \FF, \QQ, \ZZ}
  \begin{syntax}
    \cs{RR}
    \cs{CC}
    \cs{NN}
    \cs{FF}
    \cs{QQ}
    \cs{ZZ}
  \end{syntax}

  Some shorthands for common blackboard bold (\cs{mathbb}) letters are provided. These also work outside of math mode, or rather, will automatically enter math mode in that case.
\end{function}

\subsection{Vectors}

\begin{function}{\bvec}
  \begin{syntax}
    \cs{bvec} \marg{expression}
  \end{syntax}

  The \cs{bvec} command is a shorthand for \cs{mathbf} that works outside of math mode as well. It is a shorter and more semantic way to typeset vectors in bold.
\end{function}

\begin{function}{\mtx, \pmtx, \bmtx, \Bmtx, \vmtx, \Vmtx}
  \begin{syntax}
    \cs{mtx} \oarg{alignment} \marg{contents}
    \cs{pmtx} \oarg{alignment} \marg{contents}
    \cs{bmtx} \oarg{alignment} \marg{contents}
    \cs{Bmtx} \oarg{alignment} \marg{contents}
    \cs{vmtx} \oarg{alignment} \marg{contents}
    \cs{Vmtx} \oarg{alignment} \marg{contents}
  \end{syntax}

  Matrices can also be typset with a shorthand command instead of an environment; this is useful for smaller matrices. For instance:

  \[ \mtx{33 & 4 \\ 5 & 6} \]

  An optional argument can be used to specify the alignment of the columns, using the starred matrix environment variants provided in \pkg{mathtools}. For instance:

  \[ \mtx[r]{33 & 4 \\ 5 & 6} \]

  The similarly named \cs{smtx}, \cs{psmtx}, etc. refer to the small matrix variants. They likewise can take an optional argument for the alignment.
\end{function}

\section{Theorems}

The \pkg{gravy} package provides a range of colored environments for theorems, lemmas, definitions, remarks, etc. These environments are loosely color-coded by their general semantic meaning.

These environments are defined with the \pkg{thmtools} package, and thus take one optional argument which can either be a name or a key-value list of a name and a label.

\subsection{Blue}

Blue environments semantically refer to factual claims that are believed or proven to be true. The following environments, along with starred versions of each, are provided: \begin{itemize}
  \item "theorem"
  \item "lemma"
  \item "corollary"
  \item "proposition"
  \item "conjecture"
  \item "criterion"
  \item "assertion"
\end{itemize}

By default, these environments are numbered; the starred versions produce unnumbered theorems. An example of the styling is shown below.

\begin{theorem*}
  \lipsum[1][1-6]
\end{theorem*}

\subsection{Red}

Red environments semantically refer to definitions or algorithms, which by definition accurately describe their subject matter. The following environments, along with starred versions of each, are provided: \begin{itemize}
  \item "definition"
  \item "algorithm"
\end{itemize}

By default, these environments are numbered; the starred versions produce unnumbered theorems. An example of the styling is shown below.

\begin{definition*}
  \lipsum[1][1-6]
\end{definition*}

\subsection{Orange}

Orange environments semantically refer to tangential or additional information. The following environments, with no starred versions, are provided: \begin{itemize}
  \item "remark"
  \item "note"
\end{itemize}

These environments are not numbered. An example of the styling is shown below.

\begin{remark}
  \lipsum[1][1-6]
\end{remark}

\subsection{Green}

Green environments semantically refer to exercises for the reader. The following environments, along with starred versions of eaach, are provided: \begin{itemize}
  \item "example"
  \item "problem"
  \item "question"
\end{itemize}

By default, these environments are numbered; the starred versions produce unnumbered theorems. An example of the styling is shown below.

\begin{example*}
  \lipsum[1][1-6]
\end{example*}

\subsection{Proofs}

\begin{proof}
  This is a proof, created using the \env{proof} environment.
\end{proof}

\begin{proof*}
  This is a proof with no QED symbol, created using the \env{proof*} environment.
\end{proof*}

\begin{solution}
  This is a solution, created using the \env{solution} environment. Note that it has no QED symbol.
\end{solution}

\section{Vocabulary}

\begin{function}{\vocab}
  \begin{syntax}
    \cs{vocab} \marg{term}
  \end{syntax}

  Vocabulary terms can be emphasized with the \cs{vocab} command. For instance, \vocab{this} is a vocabulary word. The red color used is the same as for definitions and algorithms, and is not used for any types of links.
\end{function}

\section{Code}

Styles for code snippets are not loaded by default, and are instead loaded by passing the \texttt{minted} option to the \pkg{gravy} package or document classes. This option loads the \pkg{minted} package and defines a custom style for it. The \pkg{minted} package is not loaded if the \texttt{minted} option is not passed.

In order to function, these styles require the \texttt{pygments-gravy} custom style for the Pygments syntax highlighter. For more information and installation instructions, see the \href{https://github.com/Gravitonic/pygments-gravy}{Gravitonic/pygments-gravy} GitHub repository.

\end{document}
